\section{And-inverter graphs (AIGs)}
We will now consider another scheme of building a circuit. First we will introduce the concept of an AIG and how it can be used to predict something. Then, we will devise an initialization scheme and heuristic local search algorithm that we try out by running experiments. We will find that the local search algorithm works and we are able to iteratively improve the accuracy. The resulting AIGs are very small (way below 1 Kilobyte), so their implementation on devices with little computing power is promising. What remains to be done is to try out better search algorithms that find AIGs with higher accuracies and are faster.

\subsection{Introduction}
Let us start by introducing another predictive system that looks like a network and works with binary values. Consider binary variable $\mathsf{2}$. For consistency, which will become apparent later on, we will use numeric variables. It can either be true or false. Introducing another variable $\mathsf{4}$, we can perform a logical operation on $\mathsf{2}$ and $\mathsf{4}$, similar to how we can perform a mathematical operation with two real numbers (e.g., adding, subtracting, dividing them, etc.). Logical operators include AND ($\wedge$), OR ($\vee$), IMPLICATION ($\Rightarrow$), XOR ($\otimes$) and so on. If we compute

\begin{align} \label{eq:ceqaandb}
    \mathsf{12} = \mathsf{2} \wedge \mathsf{4},
  \end{align}where $\mathsf{12}$ is another binary variable, we can visualize this using a graph, visible in Figure~\ref{fig:aig1}. First let us consider Figure~\ref{fig:aig1-text}. We interpret variables $\mathsf{2}$ and $\mathsf{4}$ as \textit{inputs} and thus draw them using boxes and label them additionally using triangle nodes. Variable $\mathsf{6}$ in Equation~\ref{eq:ceqaandb} becomes and \textit{and-node}, i.e., it is the result of performing AND on $\mathsf{2}$ and $\mathsf{4}$, and we denote it using a circular node. Since variable $\mathsf{12}$ is already the output of equation~\ref{eq:ceqaandb}, we mark it using a triangle node. We could already imagine Equation~\ref{eq:ceqaandb} being a predictive system. Suppose we want to build a predictive system that forecasts whether or not teenage boy \enquote{Hans} will refresh himself at the family's swimming pool. We are his neighbors and have good information about his daily life, albeit limited. We observe that on a hot and sunny day, he is almost always at the pool. Thus we assign $\mathsf{2}$ to true if it is sunny and false otherwise and $\mathsf{4}$ to true if it is hot and false otherwise. Variable $\mathsf{12}$ now represents our prediction if Hans will go to the pool. Figure~\ref{fig:aig1-text} visualizes this predictive system.

\begin{figure}[!htb]
    \centering
  \begin{minipage}[b]{.3\linewidth}
    \centering
    \resizebox {0.63\textwidth} {!} {
      \includestandalone[]{standalone/aig/aig1-num}
    }
    \subcaption{Plain tree.}
    \label{fig:aig1-num}
  \end{minipage}
  \begin{minipage}[b]{.5\linewidth}
    \centering
    \resizebox {0.5\textwidth} {!} {
      \includestandalone[]{standalone/aig/aig1-text}
    }
    \subcaption{Tree with interpretation of variables.}
    \label{fig:aig1-text}
  \end{minipage}
\caption{Equation~\ref{eq:ceqaandb} visualized using a graph in (\subref{fig:aig1-num}) with a possible meaning to the variables in (\subref{fig:aig1-text}). Inputs are drawn using a box node and further labeled using triangle nodes. Circular nodes perform AND on two other nodes which can be either input nodes or other AND nodes. Outputs are denoted using triangle nodes. In (\subref{fig:aig1-num}), the same object is represented, but this time a meaning to the variables is given. We predict teenage boy Hans to go to the pool if it is sunny and hot.}
\label{fig:aig1}
\end{figure}
\FloatBarrier

\noindent Looking at Figure~\ref{fig:aig1-text}, we notice that it looks like an upside-down tree that has two branches. Consequently, given any node, nodes that are connected and above it we call \enquote{parents}. For example, variable $\mathsf{12}$ is the parent of variable $\mathsf{2}$ and $\mathsf{4}$ because it is further down in the tree. Nodes that are below a node and connected to it we call \enquote{children}, so variables $\mathsf{2}$ and $\mathsf{4}$ are children of variable $\mathsf{12}$.

\noindent We will now consider more complexity in our predictive system. We observe that on some days, Hans is mowing the lawn instead of relaxing at the pool, surely because his parents told him to do so. If binary variable $\mathsf{6}$ represents whether or not Hans has to mow the lawn, we can update our predictive system:

\begin{align} \label{eq:aig2}
  \mathsf{16} = \lnot \mathsf{6} \wedge (2 \wedge 4).
\end{align}, where the symbol $\lnot$ denotes negation and variable $\mathsf{16}$ is the new output. The visualization of Equation~\ref{eq:aig2} using a graph can be seen in Figure~\ref{fig:aig2}. There is a new input (I2) with a black dot at the connection to the next node. This black dot denotes negation.

\begin{figure}[!htb]
    \centering
  \begin{minipage}[b]{.3\linewidth}
    \centering
    \resizebox {0.875\textwidth} {!} {
      \includestandalone[]{standalone/aig/aig2-num}
    }
    \subcaption{Plain tree.}
    \label{fig:aig2-num}
  \end{minipage}
  \begin{minipage}[b]{.6\linewidth}
    \centering
    \resizebox {0.8\textwidth} {!} {
      \includestandalone[]{standalone/aig/aig2-text}
    }
    \subcaption{Tree with interpretation of variables.}
    \label{fig:aig2-text}
  \end{minipage}
\caption{Predictive system from Figure~\ref{fig:aig1} with complexity added. There is now another binary variable \enquote{Have to mow the lawn} which is an input. The black dot denotes negation. What was previously the output of the system becomes an intermediate result: \enquote{Want to go to the pool}.}
\label{fig:aig2}
\end{figure}
\FloatBarrier

\noindent Lets us now consider even more complexity. We know that Hans has a sister and sometimes we would find her alone at the pool on a hot and sunny day, but we never observe the two being together there. On one occasion, we see that Hans chases her away from the pool with his Friend. Hans' friend is not over as much as he could be and we have in vague memory that he is an avid baseball fan. We thus come up with folliwing reasoning: Hans does not like being at the pool with his sister together, probably because he thinks she is annoying. But if his friend is over, he has the courage to chase her away. If a baseball game is on TV, Hans' friend will prefer to stay at home and watch it. The resulting predictive system is visualized in Figure~\ref{fig:aig3-num} with numeric variable names and in Figure~\ref{fig:aig3} with an interpretation given to the variables. We invite the reader to take some time studying it. The upper most part can be understood easily by considering that for boolean variables $A$ and $B$: $A \vee B \Leftrightarrow \lnot (\lnot A \wedge \lnot B)$.

\begin{figure}[!htb]
    \centering
    \resizebox {0.35\textwidth} {!} {
      \includestandalone[]{standalone/aig/aig3-num}
    }
    \caption{Predictive system if teenage boy Hans will go relax himself at the family's swimming pool, with more complexity added.}
\label{fig:aig3-num}
\end{figure}
\FloatBarrier

\begin{figure}[!htb]
    \centering
    \resizebox {0.65\textwidth} {!} {
      \includestandalone[]{standalone/aig/aig3-text}
    }
    \caption{Predictive system if teenage boy Hans will go relax himself at the family's swimming pool, with more complexity added. Here we wrote an interpretation to each node. The upper most part can be understood easily by considering that for boolean variables $A$ and $B$: $A \vee B \Leftrightarrow \lnot (\lnot A \wedge \lnot B)$.}
\label{fig:aig3}
\end{figure}
\FloatBarrier

An interpretation, such as in Figure~\ref{fig:aig3} is usually not available to us. For example, in the case of the Binary-MNIST dataset, there are 784 possible inputs and a heuristic random search algorithm (that we will later introduce) searches for an AIG structure. Even if the meaning of each input is apparent to us, how these inputs are processed further down in the tree is hard to interpret. In the following sections, we will thus not try to come up with an interpretation of the trees.

Looking at Figure~\ref{fig:aig3}, it looks like we are dealing with facts and implication, but really this is a \textit{predictive system}, so its output is not guaranteed to be true. After all, we constructed this system with our own limited information. There are many more factors to consider if Hans will go to the swimming pool, such as homework, the pool being under maintemance, Hans prefering to play computer games indoor and so on. As with LUTs, a predictive system like this cannot handle uncertainty, i.e., what is the \textit{probability} of Hans going to the pool. The output is either 0 or 1. One solution would be to discretize the probability range, e.g., 0-25\%, 26-50\%, 51-75\%, 76-100\% and have a 2-bit output. In this thesis, however, we do not consider probabilistic output.

This new type of predictive systems introduced in this section are called \enquote{and-inverter graphs} (AIGs) \cite{bib:aig_wiki}. AIGs are used to represent logic functions and they can be easily implemented as circuits. We will use them in a machine learning context, i.e., given a training set with features and labels we search for an architecture that has the highest possible accuracy.

\subsection{Initialization of a random AIG}
First we must initialize a random AIG that we can later pass to the heuristic local search algorithm. We need a systematic way to build an AIG structure, where cyclicity is forbidden. We use an approach inspired from neural networks. A NN has a certain number of hidden layers and each hidden layer takes input only from the previous one. Thus there is never cyclicity. We take a similar approach for AIGs by specifying the number of hidden layers, the number of nodes per hidden layer and how many outputs there are. Each node takes two inputs (i.e., children $c_1$ and $c_2$) randomly from the previous hidden layer and the inputs to the node are randomly negated or not. Important to note is that almost all architectures initialized this way have \enquote{inactive nodes}, i.e. they do not contribute to the final prediction. Inactive nodes are still part of the architecture, though, and may become active during the search process. An AIG with all inactive nodes removed, we call a \enquote{pruned AIG}. Figure~\ref{fig:aig-init} visualizes a randomly initialized AIGs, where the inactive nodes are dotted. Algorithm~\ref{alg:aig_init} described the initialization process more formally. 

\begin{figure}[!htb]
    \centering
    \resizebox {0.45\textwidth} {!} {
      \includestandalone[]{standalone/aig/aig-init}
    }
    \caption{Randomly initialized AIG with inactive nodes (dotted). Inactive nodes do not contribute to the prediction, but may become active during the search process. Input nodes I1 and I3 also do not contribute to the prediction.}
\label{fig:aig-init}
\end{figure}
\FloatBarrier

\begin{algorithm}
  \caption{Random AIG initialization}
  \label{alg:aig_init}
  \begin{algorithmic}
    \State Given number of inputs $I$, number of hidden layers $L$, number of nodes per hidden layer $l_1, \dots, l_L$ and number of outputs $O$, construct a random AIG.
    \vspace{1em}
    \For{hidden layer $i = 1, \dots, L$}
      \For{node $l_1, \dots, l_L$}
        \State Choose $c_1$ randomly from any nodes in the previous layer
        \State Negate $c_1$ with a 50\% chance
        \State Choose $c_2$ randomly from any nodes in the previous layer,  where $c_2 \neq c_1$
        \State Negate $c_2$ with a 50\% change
      \EndFor
    \EndFor
    \For{output node $i = 1, \dots, O$}
      \State Choose $c_1$ randomly from any nodes in the last hidden layer
      \State Negate $c_1$ with a 50\% chance
      \State Choose $c_2$ randomly from any nodes in the hast hidden layer,  where $c_2 \neq c_1$
      \State Negate $c_2$ with a 50\% change
    \EndFor
  \end{algorithmic}
\end{algorithm}
\FloatBarrier

\subsection{Greedy local search}
Given our dataset and randomly initialized AIG, we would like an architecture that has the highest possible performance in terms of accuracy, meaning we need a search algorithm. We choose a greedy local search \cite{bib:koller2009probabilistic} because it is fairly simple and easy to implement, thus a great way to start. The basic idea is to consider a set of successor architectures. Such a set arises when we consider each active node in the graph and change a small thing, i.e., change a child to another node or negating a child. We score each architecture and take the one that has the highest training accuracy, thus proceeding to the next iteration. If the training accuracy does not improve for a certain number of iterations, we finish. The greedy local search is described in Algorithm~
\ref{alg:aig_local_search}.

\begin{algorithm}
  \caption{AIG greedy local search}
  \label{alg:aig_local_search}
  \begin{algorithmic}
    \State Given binary dataset $(\boldsymbol{X}, \bm{y})$ and randomly initialized AIG with architecture $\sigma_0$, perform a greedy local search.
    \vspace{1em}
    \State Choose hyperparameters:
      \Statein Patience $p$
    \State $\text{no\_change} \in \mathds{N} \gets 0$
    \State $s \in \mathds{R} \gets 0$
    \State $\sigma \gets \sigma_0$
    \While{$\text{no\_change} < p$}
      \State $\Sigma = [\hspace{0.3em}]$
      \State $S = [\hspace{0.3em}]$
      \For{all active nodes in $\sigma$}
          \State $\sigma' \gets$ change $c_1$ to any node in the previous layer
          \State $\Sigma.\text{append}(\sigma')$, $S.\text{append}(\text{accuracy\_score}(\sigma', (\bm{X}, \bm{y})))$
          \State $\sigma' \gets$ change $c_2$ to any node in the previous layer
          \State $\Sigma.\text{append}(\sigma')$, $S.\text{append}(\text{accuracy\_score}(\sigma', (\bm{X}, \bm{y})))$
          \State $\sigma' \gets$ change polarity of $c_1$
          \State $\Sigma.\text{append}(\sigma')$, $S.\text{append}(\text{accuracy\_score}(\sigma', (\bm{X}, \bm{y})))$
          \State $\sigma' \gets$ change polarity of $c_2$
          \State $\Sigma.\text{append}(\sigma')$, $S.\text{append}(\text{accuracy\_score}(\sigma', (\bm{X}, \bm{y})))$
      \EndFor
      \State $s' \gets \text{max}(S)$
      \If{s' > s}
        \State $s \gets s'$
        \State $\text{no\_change} \gets 0$
        \State $\sigma \gets \Sigma[\text{argmax}(S)]$
      \Else
        \State $\text{no\_change} \mathrel{+}= 1$
      \EndIf
    \EndWhile
  \end{algorithmic}
\end{algorithm}
\FloatBarrier

\subsection{Experiments}
We again use the Binary-MNIST dataset. We use the AIG format and library \enquote{AIGER} \cite{bib:Biere-FMV-TR-07-1} to specify and run our AIGs. The network structure and local search algorithm we have written from scratch in C++. Along with random initialization and the local search, the code features functionality to export into AIGER format and to export visualizations. The code is available on GitHub \cite{bib:aig_github}. Utilizing initialization from Algorithm~\ref{alg:aig_init} and training from Algorithm~\ref{alg:aig_local_search}, we run 42 experiments. There are 784 inputs and we choose 3 hidden layers with 64 AND nodes per hidden layer, meaning any resulting AIG will maximally have 192 AND-gates. The very low number of AND-gates that we permit is due to computational constraints, otherwise it would take too long. The output layer has one AND node and takes inputs from the last hidden layer. In Figure~\ref{fig:aig_exp_result}, 10 out of the 42 runs are visualized. The mean final training and testing accuracies of all runs 0.65 and 0.66, respectively, the minimum final accuracies are 0.51 each and 0.74 are the maxima each. The initial training accuracies range from 0.51 to 0.61. A higher initial training accuracy does not necessarily mean a higher final accuracy. A visualization of the pruned best AIG found can be seen in Appendix~\ref{app:aig-mnist}.

\begin{figure}[!htb]
    \centering
  \begin{minipage}[b]{.49\linewidth}
    \resizebox {1\textwidth} {!} {
    \includestandalone[]{standalone/aig/aig_exp_result_train}
    }
    \subcaption{Training.}
    \label{fig:aix_exp_result:train}
  \end{minipage}
  \begin{minipage}[b]{.49\linewidth}
    \resizebox {1\textwidth} {!} {
    \includestandalone[]{standalone/aig/aig_exp_result_test}
    }
    \subcaption{Testing.}
    \label{fig:aig_exp_result:test}
  \end{minipage}
  \caption{Results (training accuracies) of greedy local search according to Algorithm~\ref{alg:aig_local_search} on AIGs. We performed 38 experiments, 10 of which are visualized here. The initial training accuracies vary from 0.61 to 0.61. A higher initial training accuracy does not necessarily mean a higher final accuracy. Best of all runs was 0.77.}
\label{fig:aig_exp_result}
\end{figure}
\FloatBarrier

\noindent We can see that we are able to improve the accuracy iteratively. With AIGs, the greedy local search algorithm is probably too optimistic, though, since we quickly bump into an upper ceiling. What also happens is to get stuck at sub-optimal areas, from which the accuracy never recovers. For example, in Figure~\ref{fig:aig_exp_result}, we can see a run that starts at a very low accuracy and never goes up at all. So there is need for a different search algorithm.

The extremely small size of the AIGs is promising. With the architecture chosen for the experiment, a single AIG only has 68 Bytes of disk size (we arrived at this number by converting an ASCII-encoded AIG file to binary using AIGER and looking at the file size). A small size, together with the lack of need for floating-point matrix computations, make these architectures extremely fast and thus suitable for electronic devices with very little computing power. Furthermore, AIGs are easily translatable to circuits that are implemented on hardware, so the architecture could be much bigger.

\subsection{Summary}
We introduced AIGs and investigated whether or not they can be used as predictive systems, where we have found that this is possible. We conceptualized our own initialization scheme and chose a greedy local search as search algorithm. We wrote the code for our experiments in C++ from scratch, making use of the AIGER library. Choosing an architecture that has maximally just 192 AND-gates, the best local search run returned an AIG that has an accuracy of 0.76 which is significantly above chance, meaning learning had taken place. The extremely small size, together with the prospect of implementation on hardware make AIGs very promising. Future work should consist of trying to find better search algorithms that are able to find better architectures in less time.
