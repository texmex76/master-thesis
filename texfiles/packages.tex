\usepackage{relsize}
\usepackage[margin=3cm]{geometry} %Definiere Rand
\usepackage{graphicx} % Zum Einbinden von Bildern
\usepackage[english]{babel} % Direkte Eingabe von Umlauten
%\UseRawInputEncoding
\usepackage[T1]{fontenc}  % Direkte Eingabe von Umlauten
\usepackage{pgfplots} % Zum Einfuegen von Plots
\pgfplotsset{compat=1.14} % Damit wird beim Plotten keinen Error bekommen
\usepackage[section]{placeins} %Damit Bilder in der Section bleiben
\usepackage{amsmath} % Standard fuer mathematische Ausdruecke
\usepackage{amssymb} % Weitere Symbole
\usepackage{mathtools} % Fuer weitere mathematische Ausdruecke
\usepackage{dsfont}
\usepackage[font=small,labelfont=bf]{caption} % Kleinerer Text bei Captions
\usepackage{tabu} % Anderes Tabellenenvironment, wird am Ende fuer die Namen verwendet
\usepackage{subcaption} % Side-by-side figures with minipage
\usepackage{url} % Damit man urls zitieren kann
\usepackage[autostyle=true,german=quotes]{csquotes} % Damit Zitieren leichter ist; Bsp: \enquote{nur}
\usepackage[nottoc,numbib]{tocbibind} % Damit die Referenzen im Inhaltsverzeichnis erscheinen
\usepackage{standalone} % Zum Outsourcen von Plots
\usepackage{setspace} % Damit die naechsten funktionieren
\renewcommand{\topfraction}{0.85} % Let top 85% of a page contain a figure
\renewcommand{\textfraction}{0.1} % Default amount of minimum text on page (Set to 10%)
\renewcommand{\floatpagefraction}{0.75} % Only place figures by themselves if they take up more than 75% of the page
%
\usepackage{threeparttable, tablefootnote}
\usepackage[colorlinks,urlcolor=black,allcolors=.]{hyperref} % For hyperlinks to be clickable
\usepackage{listings} % For embedding code into the document
\usepackage{bm} % For bold italic letters in math mode, shorter than \boldsymbol
\usepackage{amsthm} % For theorems, definitions, examples, etc.
\usepackage{empheq} % For alignment inside cases
\pgfplotsset{compat=1.16} % To avoid tikz errors
\usepackage{tcolorbox} % For boxes around text
%\usepackage{enumitem} % For enumerate and itemize, custom separations
\usepackage{bbold} % For \mathbb{1}
\usepackage{xcolor, soul} % For text with color background
\definecolor{lightlightgray}{RGB}{224,224,224}
\sethlcolor{lightlightgray} % Color background color
\usepackage{algorithm} % For algorithm environment
\usepackage{algpseudocode} % For pseudocode
\newcommand{\Statein}{\State \hspace{\algorithmicindent}}

% begin vertical rule patch for algorithmicx (http://tex.stackexchange.com/questions/144840/vertical-loop-block-lines-in-algorithmicx-with-noend-option)
\makeatletter
% start with some helper code
% This is the vertical rule that is inserted
\newcommand*{\algrule}[1][\algorithmicindent]{\makebox[#1][l]{\hspace*{.5em}\vrule height .75\baselineskip depth .25\baselineskip}}%

\newcount\ALG@printindent@tempcnta
\def\ALG@printindent{%
    \ifnum \theALG@nested>0% is there anything to print
        \ifx\ALG@text\ALG@x@notext% is this an end group without any text?
            % do nothing
            \addvspace{-3pt}% FUDGE for cases where no text is shown, to make the rules line up
        \else
            \unskip
            % draw a rule for each indent level
            \ALG@printindent@tempcnta=1
            \loop
                \algrule[\csname ALG@ind@\the\ALG@printindent@tempcnta\endcsname]%
                \advance \ALG@printindent@tempcnta 1
            \ifnum \ALG@printindent@tempcnta<\numexpr\theALG@nested+1\relax% can't do <=, so add one to RHS and use < instead
            \repeat
        \fi
    \fi
    }%
\usepackage{etoolbox}
% the following line injects our new indent handling code in place of the default spacing
\patchcmd{\ALG@doentity}{\noindent\hskip\ALG@tlm}{\ALG@printindent}{}{\errmessage{failed to patch}}
\makeatother
% end vertical rule patch for algorithmicx

% Adds vertical lines to algorithm loops, for better visibility
\makeatletter
\newcommand{\algmargin}{\the\ALG@thistlm}
\makeatother
\algnewcommand{\parState}[1]{\State%
  \parbox[t]{\dimexpr\linewidth-\algmargin}{\strut #1\strut}}

% Putting this here because in the standalone it does not work
% Wasted a half day on this!
\usetikzlibrary{shapes.geometric}
\usetikzlibrary{arrows.meta, trees}
\usetikzlibrary{arrows,shapes}
\newcommand{\mySize}{\small}
\newcommand{\nodeSep}{0.75}
